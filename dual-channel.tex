\documentclass{article}
\usepackage{arxiv}
\usepackage{ctex}
\usepackage{CJK}
\usepackage{enumerate}
\usepackage[T1]{fontenc}    % use 8-bit T1 fonts
\usepackage{hyperref}       % hyperlinks
\usepackage{url}            % simple URL typesetting
\usepackage{booktabs}       % professional-quality tables
\usepackage{amsfonts}       % blackboard math symbols
\usepackage{nicefrac}       % compact symbols for 1/2, etc.
\usepackage{microtype}      % microtypography
\usepackage{lipsum}		% Can be removed after putting your text content
\usepackage{graphicx}
\usepackage{caption}
\usepackage{geometry}
\usepackage{multirow}
\usepackage{array}
\usepackage{longtable}
\usepackage{booktabs}
\usepackage{indentfirst}
\usepackage{tikz,mathpazo}
\usepackage{flowchart}
\usepackage{float}
\usepackage{subfigure}
\usepackage{mathrsfs}
\usepackage{amsfonts}
\usetikzlibrary{arrows,shapes,chains}
\setlength{\parindent}{2em}
\setlength{\abovecaptionskip}{0cm}
\setlength{\belowcaptionskip}{-0.25cm}
\title{双渠道绿色供应链博弈模型的稳定性研究}
\author{
    吴俊达\\数学与统计学院\\统计与精算系\\joshua19801010@gmail.com\\
    \And
    杨帆\\数学与统计学院\\应用数学系\\fany02656@gmail.com
    \And
    刘芝延\\数学与统计学院\\应用数学系\\
}

\begin{document}
\maketitle
\begin{abstract}

\end{abstract}
\keywords{}
\section{介绍}
\par \textbf{
    问题的研究背景(说明研究的问题的背景是什么,给出近年来该问题在学术领域的发展情况)。
    问题来源(如果没有问题存在的意义,那就没有必要写论文了)。
    note: 双渠道供应链背景(电商、厂商),关于经济学的定性研究结论。
}
\par 传统的双渠道供应链模型包括一个制造商和一个传统经销商的博弈模式。然后近年来许多学者从实际生产生活中抽象出多种模型,并分别进行了建模和均衡的求解。Wang等(2016)\cite{2016Wang}研究了厂商增加电商和第三方直销这两种额外渠道,分别对传统博弈结果的影响。Zhao(2017)\cite{2017Zhao}等人求解了两个制造商和一个经销商的情况。Xie(2018)\cite{2018Xie}等人综合前人研究增加了厂商和经销商的一致模式,并给出了完整的数学求解。Arshad(2018)\cite{2018Arshad}等人讨论了两个经销商的合作对抗厂商的博弈策略,并加入商品回收机制,考虑闭环供应链。Pathak、Giri和Hosseini-Motlagh等(2019)\cite{2019Pathak}\cite{2019Giri,2019Hosseini-Motlagh}都研究了闭环供应链模型中回收商的组织和回收价格对博弈策略的作用,并且讨论了参数对于均衡点稳定性的影响。\textbf{note:经销商的公平竞争考量,现实意义和措施。}王虹等(2009)\cite{2009Wang}讨论了一致与非一致下的Stackelberg博弈模型,认为厂商的决策在非一致模式下优于一致模式。Li等(2016)\cite{2016Li}研究了非一致模式下考虑公平因素的Stackelberg博弈,认为经销商会在已知不对等的地位下,通过降低价格来获取市场份额的增加。Wang(2018)\cite{2018Wang}等人讨论了一致模式和非一致Bertrand模式下引入公平考虑的作用。Dai等(2019)\cite{2019Dai}进一步进行了同样的建模求解,进一步给出了参数显著性的分析。
\par 这一类动态博弈模型,以及其演化模式,通常描述为一个动力系统模型。因此运用动力学分岔理论研究博弈系统的稳定性成为主流研究方法。唐兴巧等(2017)\cite{2017Tang}利用数值模拟系统的分岔图等动力学特征分析了了系统Nash均衡点的稳定性,并得出了系统存在混沌行为导致市场不规律的后果。张芳等(2018)\cite{2018Zhang}和Zhang等(2015)\cite{2015Zhang}分别在公平因素条件下和闭环系统条件下,利用同样的工具研究动力系统的Hopf分岔图、Lyapunov指数以及混沌引子等指标。Li等人在2018年,通过一系列问题的实证研究,将该方法运用到了各个博弈场景中\cite{2018Huang,2018Qiu-Xiang,2019Qiu-Xiang1,2019Qiu-Xiang2}。
\par \textbf{note: 绿色供应链和制度约束(ECSC)\cite{2005Beamon}。}Sheu等(2012)\cite{2012Sheu}的研究发现,当政府基于环保因素考虑,通过财政手段对供应链主体征税以及补贴,供应链博弈均衡的收益和社会福利都有显著的提高。进一步,Sinayi等(2018)\cite{2018Sinayi}和Aslani等(2019)\cite{2019Aslani}引入了消费者对绿色产品的线性倾向性需求,同时考虑制造商生产绿色产品的额外成本,研究政府的各类干预措施分别的有效性。Rahmani等(2018)\cite{2018Rahami}研究了需求扰动下的双渠道绿色供应链的定价模式。
\par 本文的工作主要分为三个部分:模型的建立、稳定性分析、得出的结论。在第二节,我们主要完成了传统双渠道供应链Stackelberg博弈模型,以及其动力系统模型的建立。第三节中我们采用了主流文献中讨论的,基于环保税的线性惩罚项(主要形式为征税和补贴),以及厂商主动额外附加的生产环保产品的二次项成本,对基础的博弈模型和动力系统模型进行修改。第四节中,我们分别对二、三节中模型进行均衡点求解。进而,我们利用数值方法,完成了基于环保参数的演化分析,给出度量混沌效应的Hopf分岔图和最大Lyapunov指数图。在这一节中,我们验证了政府基于环保考虑对博弈均衡以及供应链收益的影响。同时,通过对比制度约束下系统稳定性状况,试图阐述政府对该类型供应链的制度监管对行业发展健康程度的影响。最后一节,我们将从经济和社会的层面,试图重新解释我们得出的结果。本文的创新点主要在以下:%电动汽车补贴
\begin{itemize}
    \item 在绿色供应链的研究中,文献主要集中于成本、固定税率以及绿色产品的受欢迎程度上。我们引入了与绿色程度线性相关的政府补贴因素,作为绿色产品的附加竞争力描述。
    \item 我们完成了双渠道绿色供应链的建模求解,同时创新地分析了环保制度对于供应链的稳定性影响。在经济和社会的层面,我们的研究可以一定程度下填补制度设计对市场稳定性考虑的不足。
    \item 在系统混沌分析方面的工作,我们主要依赖于数值模拟的研究方法,该方法对于模型表达式的微分性质等要求较低。同时在计算Lyapunov指数的工作中,我们不同于大多数论文中基于特征矩阵(Jacobi矩阵)的解析算法,利用了其原始定义,并通过大样本模拟迭代的基础下完成求解。该方法易于推广到更复杂的,以及无法显式求解的模型中去。
    \item 在数值计算方面的工作,我们不再依赖常用的matlab工具包。我们编写了针对三个博弈目标(可非常容易地扩展到多个)的动力系统混沌分析python工具包,计算Hopf分岔轨迹以及模拟计算Lyapunov指数。基于混沌分析模拟的可并发性,我们主要支持基于cuda计算库和Nvidia GPU的高速计算。实验在GeForce 980Ti的环境下,完成了每秒约$2\times10^5$个样本,每个样本1000次的迭代计算。(GitHub: JoshuaWu1997/chaos-analysis.git)
\end{itemize}
\section{问题描述与模型建立}
\par 本文考虑简单的双渠道供应链模型,其中包含一个供应商和一个一级零售商。供应商生产商品,分别通过批发给零售商以及通过直销渠道卖给顾客获利。零售商则从供应商进货,转卖给顾客获利。顾客则存在直销和进销商两个渠道购得商品。我们通过对市场的基本假设,固定了二者无需博弈的因素。同时我们引入了第三方监管力度的参量。我们的建模如下:
\par \textbf{基本假设一:}我们考虑线性的供需关系,其中固有需求$\alpha$边际需求与边际替代率假设相同$\beta$,即认为两种渠道的商品完全一致,没有差别,并且直销价格和经销商售价一致为$p$\cite{2016Li}。在绿色供应链中,假设只存在一种绿色商品,其绿色程度$\theta$系数直接影响受到顾客额外的欢迎程度和生产成本\cite{2012Ghosh}。从而,该统一价格的双方市场需求分别为:
\begin{equation}
    D_r=\alpha-\beta p+\lambda_r\theta
\end{equation}
\begin{equation}
    D_m=\alpha-\beta p+\lambda_m\theta
\end{equation}
\par 其中$\lambda_r, \lambda_m$分别度量了消费者对不同渠道相同绿色产品的不同认同度。
\par \textbf{基本假设二:}我们考虑政府对拥有不同$\theta$的商品进行价格补贴$\tau$。Sinayi(2018)也在其模型中加入了一次性的影响,即附加一个固定的价格补贴\cite{2018Sinayi}。基于现实政策中,逐渐出现更为灵活的,基于绿色程度的奖惩制度,于是我们创新地引入了与$\theta$线性相关的补贴项。由于方程的线性性质,该影响又可以被看作是在消费者角度,通过价格优势形成的额外需求。从而与消费者主观偏好系数$\lambda$可被统一归因为同类型影响,即:
\begin{equation}
    D_r=\alpha-\beta (p-\tau\theta)+\lambda_r\theta=\alpha-\beta p+(\lambda_r+\tau)\theta
\end{equation}
\begin{equation}
    D_m=\alpha-\beta (p-\tau\theta)+\lambda_m\theta=\alpha-\beta p+(\lambda_m+\tau)\theta
\end{equation}
\par \textbf{基本假设三:}我们假设制造商制造绿色商品的额外成本与$\frac{\theta^2}{2}$相关,系数$\eta$。制造商批发给零售商的价格$w$低于售价$p$,高于其基础成本$c$。则二者的收益(效用)可被表示为\cite{2012Ghosh}\cite{2016Li}:
\begin{equation}
    U_m(p,w,\theta)=\pi_m=(w-c)(\alpha-\beta p+\lambda_r\theta)+(p-c)(\alpha-\beta p+\lambda_m\theta)-\eta\frac{\theta^2}{2}
\end{equation}
\begin{equation}
    U_r(p,w,\theta)=\pi_r=(p-w)(\alpha-\beta p+\lambda_r\theta)
\end{equation}
\par \textbf{基本假设四:}基于Stackelberg博弈模型,该供应链的批发价格$w$以及商品的绿色程度$\theta$先由供应商确定。其后,零售商通过调整销售价格来达到自己利益的最大化\cite{2009Wang},此时基于假设一,直销渠道价格必须跟随。同时,政府可以通过调控价格补贴参数$\lambda$和成本系数$\eta$来调控市场。
\section{均衡求解}
\par 在基于Stackelberg假设下,供应商首先做出反应,供应商可以同时调整批发价格$w$和绿色程度$\theta$来使得自己的利益最大化。在给定的市场价格$p$的情况下,供应商的边际效用(效用函数的梯度)为:
\begin{equation}
    \frac{\partial U_m}{\partial w}(w;p,\theta)=\alpha-\beta p+\lambda_r\theta
\end{equation}
\begin{equation}
    \frac{\partial U_m}{\partial \theta}(\theta;p,w)=(w-c)\lambda_r+(p-c)\lambda_m-\eta\theta
\end{equation}
\par 同理可得零售商对于供应商给出的$w,\theta$做出反应,调整市场价格,其边际效益为:
\begin{equation}
    \frac{\partial U_m}{\partial p}(p;w,\theta)=\alpha-\beta (2p-w)+\lambda_r\theta
\end{equation}
\par 在实际博弈场景下,由于有限理性的约束,博弈双方只能在当前状态下做出能力范围内的调整,是的自己的收益向更优的方向发展。此时,假设博弈目标$p, w, \theta$相对应的每次调整能力固定为$k_p, k_w, k_\theta$,则可构建动态博弈模型,即时间离散的动力系统:
\begin{equation}  
    \left\{  
        \begin{array}{lr}
            w(t+1)=w(t)+k_ww(t)[\alpha-\beta p(t)+\lambda_r\theta(t)]               \\  
            p(t+1)=p(t)+k_pp(t)[\alpha-2\beta p(t)+\lambda_r\theta(t)+\beta w(t)]   \\  
            \theta(t+1)=\theta(t)+k_\theta\theta(t)[\lambda_rw(t)+\lambda_mp(t)-\eta\theta(t)-c(\lambda_r+\lambda_m)]
        \end{array}  
    \right.
\end{equation}  
\par 对于Stackelberg博弈,零售商只能被动的接受厂商事先设定的$w, \theta$,于是对于零售商,其定价的均衡解为边际效用为零时:
\begin{equation}
    p^*(w,\theta)=\frac{\alpha+\lambda_r\theta+\beta w}{2\beta}
\end{equation}
\par 此时的厂商的效益最大化,可看作厂商内部的最优化问题,因为市场价格与自身生产的关系已知,将式(11)带入式(5)得到到厂商效益:
\begin{equation}
    U_m^*(w,\theta)=\frac{1}{2}(w-c)(\alpha+\lambda_r\theta-\beta w)+\frac{1}{2}(\alpha+2\lambda_m\theta-\lambda_r\theta-\beta w)(\frac{\alpha+\lambda_r\theta+\beta w}{2\beta}-c)-\eta\frac{\theta^2}{2}
\end{equation}
\section{稳定性的数值分析}
\par 数值分析模拟动力系统演化模式是一种非常直观有效的工具。在本文所描述的系统中,我们重点关注外部影响,包括政府对绿色原料的补贴($\eta$),以及对绿色产品的支持力度($\lambda_r, \lambda_m$),对双方博弈路径的影响,以及其中可能产生的混沌现象。而我们研究的目标即为双方的博弈目标:批发价$w$,零售价$p$和绿色等级$\theta$。对动力系统混沌性的描述,我们采用了文献中普遍采用的二维分岔图,以及Lyapunov指数\cite{2017Tang}\cite{2018Zhang}\cite{2015Zhang}\cite{2018Huang}。除上述研究因素以外,我们不失一般性的对余下的参数预先赋值,简化模型计算成本,同时提出以下假设:
\par \textbf{基本假设五:}在以下的模拟计算中,我们假设$\alpha=3, \beta=1, c=1$。我们认为绿色产品在直销渠道的吸引力较弱于经销渠道,假设这种差异为一个较小的常数(否则二者竞争差异过大,系统在很早的时候就会发生崩溃),即$\lambda_r=\lambda_m+d$,通过实验结果我们取$d=0.12$较为适中。
\par 不同于现有论文基于Matlab计算的研究成果,我们设计开源了基于CUDA并发的数值模拟工具chaos-analysis (github.com/JoshuaWu1997/chaos-analysis.git)。基于该模拟过程高度相似的简单迭代,以及所需的样本数量庞大的特性,我们设计了利用GPU硬件的并发求解算法。该算法适用于广泛的多变量动力系统问题中,且对表达式的要求较低,可以扩展到大量复杂的问题中。以下的实验,我们模拟的规模达到了对参数区间进行2000-4000次的划分,每个分割点进行1000-2000次随机初值的实验,每次实验进行1000次的迭代,保证实验能完整覆盖样本空间,同时迭代到均衡点。
\par 我们同样也采用模拟的方式完成Lyapunov指数的计算。
\par 基于以上的数值分析算法,我们对前文的模型进行以下几个角度的讨论和分析,同时结合实际现象和经济学知识对所发现的数学规律进行解释,从而对于一系列现实问题提出一种角度的诠释。
\subsection{讨论一:$k_w=k_p$情形下系统对$\eta$和$\lambda$的反馈}
\par 
\section{结论}
\par

\begin{thebibliography}{1}
\bibitem{2005Beamon} Beamon, B.M. SCI ENG ETHICS (2005) 11: 221. https://doi.org/10.1007/s11948-005-0043-y
\bibitem{2009Wang} 王虹, 周晶. 不同价格模式下的双渠道供应链决策研究[J]. 中国管理科学, 2009, V17(6):84-90.
\bibitem{2012Sheu} Jiuh-Biing Sheu, Yenming J. Chen, Impact of government financial intervention on competition among green supply chains, \emph{International Journal of Production Economics}, Volume 138, Issue 1, 2012, Pages 201-213, ISSN 0925-5273, https://doi.org/10.1016/j.ijpe.2012.03.024.
\bibitem{2012Ghosh} Debabrata Ghosh, Janat Shah, A comparative analysis of greening policies across supply chain structures, \emph{International Journal of Production Economics}, Volume 135, Issue 2, 2012, Pages 568-583, ISSN 0925-5273, https://doi.org/10.1016/j.ijpe.2011.05.027.
\bibitem{2015Zhang} Fang Zhang, Junhai Ma, Research on the complex features about a dual-channel supply chain with a fair caring retailer, \emph{Communications in Nonlinear Science and Numerical Simulation}, Volume 30, Issues 1–3, 2016, Pages 151-167, ISSN 1007-5704, https://doi.org/10.1016/j.cnsns.2015.06.009.
\bibitem{2016Wang} Cong Wang, De-li Yang, and Zhao Wang, “Comparison of Dual-Channel Supply Chain Structures: E-Commerce Platform as Different Roles,” \emph{Mathematical Problems in Engineering}, vol. 2016, Article ID 3831624, 10 pages, 2016. https://doi.org/10.1155/2016/3831624.
\bibitem{2016Li} Qing-Hua Li, Bo Li, Dual-channel supply chain equilibrium problems regarding retail services and fairness concerns, \emph{Applied Mathematical Modelling}, Volume 40, Issues 15–16, 2016, Pages 7349-7367, ISSN 0307-904X, https://doi.org/10.1016/j.apm.2016.03.010.
\bibitem{2017Zhao} Jing Zhao, Xiaorui Hou, Yunlian Guo, Jie Wei, Pricing policies for complementary products in a dual-channel supply chain, \emph{Applied Mathematical Modelling}, Volume 49, 2017, Pages 437-451, ISSN 0307-904X, https://doi.org/10.1016/j.apm.2017.04.023.
\bibitem{2017Tang} 唐兴巧, 蒲新会, 蔡成松. 一类双渠道供应链博弈模型的分岔与混沌分析[J]. 温州大学学报(自然科学版), 2017(2).
\bibitem{2018Xie} Xie Fengqin, Nie Qixuan, Yu Hongfei, Inventory Strategy of Dual-Channel Supply Chain from Manufacturer's Perspective, \emph{International Journal of Management and Fuzzy Systems}. Vol. 4, No. 2, 2018, pp. 29-34. doi: 10.11648/j.ijmfs.20180402.14
\bibitem{2018Arshad} Arshad, M.; Khalid, Q.S.; Lloret, J.; Leon, A. An Efficient Approach for Coordination of Dual-Channel Closed-Loop Supply Chain Management. \emph{Sustainability} 2018, 10, 3433.
\bibitem{2018Wang} Yuyan Wang, Zhaoqing Yu , Liang Shen (2019) Study on the decision-making and coordination of an e-commerce supply chain with manufacturer fairness concerns, \emph{International Journal of Production Research}, 57:9, 2788-2808, DOI: 10.1080/00207543.2018.1500043
\bibitem{2018Li} 李盈盈,张雪梅,陈媛媛等.不同价格模式下的双渠道供应链决策研究[J].阜阳师范学院学报(自然科学版),2018,35(3):21-27.DOI:10.14096/j.cnki.cn34-1069/n/1004-4329(2018)03-021-07.
\bibitem{2018Zhang} 张芳, 马小林. 双渠道闭环供应链博弈模型的复杂性分析[J]. 天津工业大学学报, 2018, v.37;No.180(03):78-84.
\bibitem{2018Huang} Huang Yi-min, Li Qiu-xiang, and Zhang Yu-hao, “The Complexity Analysis for Price Game Model of Risk-Averse Supply Chain Considering Fairness Concern,” \emph{Complexity}, vol. 2018, Article ID 9216193, 15 pages, 2018. https://doi.org/10.1155/2018/9216193.
\bibitem{2018Qiu-Xiang} Li Qiu-xiang, Zhang Yu-hao, and Huang Yi-min, “The Complexity Analysis in Dual-Channel Supply Chain Based on Fairness Concern and Different Business Objectives,” \emph{Complexity}, vol. 2018, Article ID 4752765, 13 pages, 2018. https://doi.org/10.1155/2018/4752765.
\bibitem{2018Sinayi} Mohammadreza Sinayi, Morteza Rasti-Barzoki, A game theoretic approach for pricing, greening, and social welfare policies in a supply chain with government intervention, \emph{Journal of Cleaner Production}, Volume 196, 2018, Pages 1443-1458, ISSN 0959-6526, https://doi.org/10.1016/j.jclepro.2018.05.212.
\bibitem{2018Rahami} Kavian Rahmani, Mohammad Yavari, Pricing policies for a dual-channel green supply chain under demand disruptions, \emph{Computers \& Industrial Engineering}, Volume 127, 2019, Pages 493-510, ISSN 0360-8352, https://doi.org/10.1016/j.cie.2018.10.039.
\bibitem{2019Pathak} Pathak, U., Kant, R. , Shankar, R. OPSEARCH (2019). https://doi.org/10.1007/s12597-019-00421-z
\bibitem{2019Giri} B.C. Giri, S.K. Dey, Game theoretic analysis of a closed-loop supply chain with backup supplier under dual channel recycling, \emph{Computers \& Industrial Engineering}, Volume 129, 2019, Pages 179-191, ISSN 0360-8352, https://doi.org/10.1016/j.cie.2019.01.035. 
\bibitem{2019Hosseini-Motlagh} Seyyed-Mahdi Hosseini-Motlagh, Mina Nouri-Harzvili, Tsan-Ming Choi, Samira Ebrahimi, Reverse supply chain systems optimization with dual channel and demand disruptions: Sustainability, CSR investment and pricing coordination, \emph{Information Sciences}, Volume 503, 2019, Pages 606-634, ISSN 0020-0255, https://doi.org/10.1016/j.ins.2019.07.021.
\bibitem{2019Dai} Lufeng Dai, Xifu Wang, Xiaoguang Liu, and Lai Wei, “Pricing Strategies in Dual-Channel Supply Chain with a Fair Caring Retailer,” \emph{Complexity}, vol. 2019, Article ID 1484372, 23 pages, 2019. https://doi.org/10.1155/2019/1484372.
\bibitem{2019Qiu-Xiang1} Li, Q.; Chen, X.; Huang, Y. The Stability and Complexity Analysis of a Low-Carbon Supply Chain Considering Fairness Concern Behavior and Sales Service. \emph{Int. J. Environ. Res. Public Health} 2019, 16, 2711.
\bibitem{2019Qiu-Xiang2} Li, Q.; Shi, M.; Deng, Q.; Huang, Y.-M. The Complexity Entropy Analysis of a Supply Chain System Considering Recovery Rate and Channel Service. \emph{Entropy} 2019, 21, 659.
\bibitem{2019Aslani} Amin Aslani, Jafar Heydari, Transshipment contract for coordination of a green dual-channel supply chain under channel disruption, \emph{Journal of Cleaner Production}, Volume 223, 2019, Pages 596-609, ISSN 0959-6526, https://doi.org/10.1016/j.jclepro.2019.03.186.

\end{thebibliography}
\end{document}