\documentclass{article}
\usepackage{arxiv}
\usepackage{ctex}
\usepackage{CJK}
\usepackage{enumerate}
\usepackage[T1]{fontenc}    % use 8-bit T1 fonts
\usepackage{hyperref}       % hyperlinks
\usepackage{url}            % simple URL typesetting
\usepackage{booktabs}       % professional-quality tables
\usepackage{amsfonts}       % blackboard math symbols
\usepackage{nicefrac}       % compact symbols for 1/2, etc.
\usepackage{microtype}      % microtypography
\usepackage{lipsum}		% Can be removed after putting your text content
\usepackage{graphicx}
\usepackage{caption}
\usepackage{geometry}
\usepackage{multirow}
\usepackage{array}
\usepackage{longtable}
\usepackage{booktabs}
\usepackage{indentfirst}
\usepackage{tikz,mathpazo}
\usepackage{flowchart}
\usepackage{float}
\usepackage{subfigure}
\usepackage{mathrsfs}
\usepackage{amsfonts}
\usetikzlibrary{arrows,shapes,chains}
\setlength{\parindent}{2em}
\setlength{\abovecaptionskip}{0cm}
\setlength{\belowcaptionskip}{-0.25cm}
\title{约束激励机制对双渠道供应链博弈稳定性的研究}
\author{
    吴俊达\\数学与统计学院\\统计与精算系\\joshua19801010@gmail.com\\
    \And
    杨帆\\数学与统计学院\\应用数学系\\fany02656@gmail.com
    \And
    刘芝延\\数学与统计学院\\应用数学系\\
}

\begin{document}
\maketitle
\begin{abstract}

\end{abstract}
\keywords{}
\section{介绍}
\par 1、问题的研究背景(说明研究的问题的背景是什么,给出近年来该问题在学术领域的发展情况)
\par 2、问题来源(如果没有问题存在的意义,那就没有必要写论文了。)
\par note: 双渠道供应链背景(电商、厂商),关于经济学的定性研究结论。
\par 传统的双渠道供应链模型包括一个制造商和一个传统经销商的博弈模式。然后今年来许多学者从实际生产生活中抽象出多种模型,并分别进行了建模和均衡的求解。Wang等(2016)\cite{2016Wang}研究了厂商增加电商和第三方直销这两种额外渠道,分别对传统博弈结果的影响。Zhao(2017)\cite{2017Zhao}等人求解了两个制造商和一个经销商的情况。Xie(2018)\cite{2018Xie}等人综合前人研究增加了厂商和经销商的合作博弈,并给出了完整的数学求解。Arshad(2018)\cite{2018Arshad}等人讨论了两个经销商的合作对抗厂商的博弈策略,并加入商品回收机制,考虑闭环供应链。Pathak、Giri和Hosseini-Motlagh等(2019)\cite{2019Pathak}\cite{2019Giri}\cite{2019Hosseini-Motlagh}都研究了闭环供应链模型中回收商的组织和回收价格对博弈策略的作用,并且讨论了参数对于均衡点稳定性的影响。

\par 3、提出的方法(这部分就是重点了,列出论文的贡献点,写清楚你的创作点,突出自己的方法与别人的不一样的地方。)
\par 4、实验结论(只有当你论文中有一个强有力的结论时才添加,这个部分需要谨慎)
\par 5、相关工作

\par 6、论文概要

\begin{thebibliography}{1}
\bibitem{2016Wang} Cong Wang, De-li Yang, and Zhao Wang, “Comparison of Dual-Channel Supply Chain Structures: E-Commerce Platform as Different Roles,” \emph{Mathematical Problems in Engineering}, vol. 2016, Article ID 3831624, 10 pages, 2016. https://doi.org/10.1155/2016/3831624.
\bibitem{2017Zhao} Jing Zhao, Xiaorui Hou, Yunlian Guo, Jie Wei, Pricing policies for complementary products in a dual-channel supply chain, \emph{Applied Mathematical Modelling}, Volume 49, 2017, Pages 437-451, ISSN 0307-904X, https://doi.org/10.1016/j.apm.2017.04.023.
\bibitem{2018Xie} Xie Fengqin, Nie Qixuan, Yu Hongfei, Inventory Strategy of Dual-Channel Supply Chain from Manufacturer's Perspective, \emph{International Journal of Management and Fuzzy Systems}. Vol. 4, No. 2, 2018, pp. 29-34. doi: 10.11648/j.ijmfs.20180402.14
\bibitem{2018Arshad} Arshad, M.; Khalid, Q.S.; Lloret, J.; Leon, A. An Efficient Approach for Coordination of Dual-Channel Closed-Loop Supply Chain Management. \emph{Sustainability} 2018, 10, 3433.
\bibitem{2019Pathak} Pathak, U., Kant, R. , Shankar, R. OPSEARCH (2019). https://doi.org/10.1007/s12597-019-00421-z
\bibitem{2019Giri} B.C. Giri, S.K. Dey, Game theoretic analysis of a closed-loop supply chain with backup supplier under dual channel recycling, \emph{Computers \& Industrial Engineering}, Volume 129, 2019, Pages 179-191, ISSN 0360-8352, https://doi.org/10.1016/j.cie.2019.01.035. 
\bibitem{2019Hosseini-Motlagh} Seyyed-Mahdi Hosseini-Motlagh, Mina Nouri-Harzvili, Tsan-Ming Choi, Samira Ebrahimi, Reverse supply chain systems optimization with dual channel and demand disruptions: Sustainability, CSR investment and pricing coordination, \emph{Information Sciences}, Volume 503, 2019, Pages 606-634, ISSN 0020-0255, https://doi.org/10.1016/j.ins.2019.07.021.
\end{thebibliography}
\end{document}