\documentclass{article}
\usepackage{arxiv}
\usepackage{ctex}
\usepackage{CJK}
\usepackage{enumerate}
\usepackage[T1]{fontenc}    % use 8-bit T1 fonts
\usepackage{hyperref}       % hyperlinks
\usepackage{url}            % simple URL typesetting
\usepackage{booktabs}       % professional-quality tables
\usepackage{amsfonts}       % blackboard math symbols
\usepackage{nicefrac}       % compact symbols for 1/2, etc.
\usepackage{microtype}      % microtypography
\usepackage{lipsum}		% Can be removed after putting your text content
\usepackage{graphicx}
\usepackage{caption}
\usepackage{geometry}
\usepackage{multirow}
\usepackage{array}
\usepackage{longtable}
\usepackage{booktabs}
\usepackage{indentfirst}
\usepackage{tikz,mathpazo}
\usepackage{flowchart}
\usepackage{float}
\usepackage{subfigure}
\usepackage{mathrsfs}
\usepackage{amsfonts}
\usetikzlibrary{arrows,shapes,chains}
\setlength{\parindent}{2em}
\setlength{\abovecaptionskip}{0cm}
\setlength{\belowcaptionskip}{-0.25cm}
\title{(*约束)对双渠道供应链博弈模型稳定性的研究}
\author{
    吴俊达\\数学与统计学院\\统计与精算系\\joshua19801010@gmail.com\\
    \And
    杨帆\\数学与统计学院\\应用数学系\\fany02656@gmail.com
    \And
    刘芝延\\数学与统计学院\\应用数学系\\
}

\begin{document}
\maketitle
\begin{abstract}

\end{abstract}
\keywords{}
\section{介绍}
\par \textbf{
    问题的研究背景(说明研究的问题的背景是什么,给出近年来该问题在学术领域的发展情况)。
    问题来源(如果没有问题存在的意义,那就没有必要写论文了)。
    note: 双渠道供应链背景(电商、厂商),关于经济学的定性研究结论。
}
\par 传统的双渠道供应链模型包括一个制造商和一个传统经销商的博弈模式。然后今年来许多学者从实际生产生活中抽象出多种模型,并分别进行了建模和均衡的求解。Wang等(2016)\cite{2016Wang}研究了厂商增加电商和第三方直销这两种额外渠道,分别对传统博弈结果的影响。Zhao(2017)\cite{2017Zhao}等人求解了两个制造商和一个经销商的情况。Xie(2018)\cite{2018Xie}等人综合前人研究增加了厂商和经销商的一致模式,并给出了完整的数学求解。Arshad(2018)\cite{2018Arshad}等人讨论了两个经销商的合作对抗厂商的博弈策略,并加入商品回收机制,考虑闭环供应链。Pathak、Giri和Hosseini-Motlagh等(2019)\cite{2019Pathak}\cite{2019Giri,2019Hosseini-Motlagh}都研究了闭环供应链模型中回收商的组织和回收价格对博弈策略的作用,并且讨论了参数对于均衡点稳定性的影响。\textbf{note:经销商的公平竞争考量,现实意义和措施。}王虹等(2009)\cite{2009Wang}讨论了一致与非一致下的Stackelberg博弈模型,认为厂商的决策在非一致模式下优于一致模式。Li等(2016)\cite{2016Li}研究了非一致模式下考虑公平因素的Stackelberg博弈,认为经销商会在已知不对等的地位下,通过降低价格来获取市场份额的增加。Wang(2018)\cite{2018Wang}等人讨论了一致模式和非一致Bertrand模式下引入公平考虑的作用。Dai等(2019)\cite{2019Dai}进一步进行了同样的建模求解,进一步给出了参数显著性的分析。
\par \textbf{note: 双渠道供应链博弈模型,由于(有限理性、有限信息、多阶段博弈)通常被建模为一个动力学微分方程系统。}因此运用动力学分岔理论研究博弈系统的稳定性成为主流研究方法。唐兴巧等(2017)\cite{2017Tang}利用数值模拟系统的分岔图等动力学特征分析了了系统Nash均衡点的稳定性,并得出了系统存在混沌行为导致市场不规律的后果。张芳等(2018)\cite{2018Zhang}和Zhang等(2015)\cite{2015Zhang}分别在公平因素条件下和闭环系统条件下,利用同样的工具研究动力系统的Hopf分岔图、Lyapunov指数以及混沌引子等指标。Li等人在2018年,通过一系列问题的实证研究,将该方法运用到了各个博弈场景中。\cite{2018Huang,2018Qiu-Xiang,2019Qiu-Xiang1,2019Qiu-Xiang2}
\par \textbf{note: 绿色经济和制度约束}
\par \textbf{
    提出的方法(这部分就是重点了,列出论文的贡献点,写清楚你的创作点,突出自己的方法与别人的不一样的地方)。
    实验结论(只有当你论文中有一个强有力的结论时才添加,这个部分需要谨慎)。
    相关工作。
}
\par 6、论文概要

\begin{thebibliography}{1}
\bibitem{2009Wang} 王虹, 周晶. 不同价格模式下的双渠道供应链决策研究[J]. 中国管理科学, 2009, V17(6):84-90.
\bibitem{2015Zhang} Fang Zhang, Junhai Ma, Research on the complex features about a dual-channel supply chain with a fair caring retailer, \emph{Communications in Nonlinear Science and Numerical Simulation}, Volume 30, Issues 1–3, 2016, Pages 151-167, ISSN 1007-5704, https://doi.org/10.1016/j.cnsns.2015.06.009.
\bibitem{2016Wang} Cong Wang, De-li Yang, and Zhao Wang, “Comparison of Dual-Channel Supply Chain Structures: E-Commerce Platform as Different Roles,” \emph{Mathematical Problems in Engineering}, vol. 2016, Article ID 3831624, 10 pages, 2016. https://doi.org/10.1155/2016/3831624.
\bibitem{2016Li} Qing-Hua Li, Bo Li, Dual-channel supply chain equilibrium problems regarding retail services and fairness concerns, \emph{Applied Mathematical Modelling}, Volume 40, Issues 15–16, 2016, Pages 7349-7367, ISSN 0307-904X, https://doi.org/10.1016/j.apm.2016.03.010.
\bibitem{2017Zhao} Jing Zhao, Xiaorui Hou, Yunlian Guo, Jie Wei, Pricing policies for complementary products in a dual-channel supply chain, \emph{Applied Mathematical Modelling}, Volume 49, 2017, Pages 437-451, ISSN 0307-904X, https://doi.org/10.1016/j.apm.2017.04.023.
\bibitem{2017Tang} 唐兴巧, 蒲新会, 蔡成松. 一类双渠道供应链博弈模型的分岔与混沌分析[J]. 温州大学学报(自然科学版), 2017(2).
\bibitem{2018Xie} Xie Fengqin, Nie Qixuan, Yu Hongfei, Inventory Strategy of Dual-Channel Supply Chain from Manufacturer's Perspective, \emph{International Journal of Management and Fuzzy Systems}. Vol. 4, No. 2, 2018, pp. 29-34. doi: 10.11648/j.ijmfs.20180402.14
\bibitem{2018Arshad} Arshad, M.; Khalid, Q.S.; Lloret, J.; Leon, A. An Efficient Approach for Coordination of Dual-Channel Closed-Loop Supply Chain Management. \emph{Sustainability} 2018, 10, 3433.
\bibitem{2018Wang} Yuyan Wang, Zhaoqing Yu , Liang Shen (2019) Study on the decision-making and coordination of an e-commerce supply chain with manufacturer fairness concerns, \emph{International Journal of Production Research}, 57:9, 2788-2808, DOI: 10.1080/00207543.2018.1500043
\bibitem{2018Li} 李盈盈,张雪梅,陈媛媛等.不同价格模式下的双渠道供应链决策研究[J].阜阳师范学院学报(自然科学版),2018,35(3):21-27.DOI:10.14096/j.cnki.cn34-1069/n/1004-4329(2018)03-021-07.
\bibitem{2018Zhang} 张芳, 马小林. 双渠道闭环供应链博弈模型的复杂性分析[J]. 天津工业大学学报, 2018, v.37;No.180(03):78-84.
\bibitem{2018Huang} Huang Yi-min, Li Qiu-xiang, and Zhang Yu-hao, “The Complexity Analysis for Price Game Model of Risk-Averse Supply Chain Considering Fairness Concern,” \emph{Complexity}, vol. 2018, Article ID 9216193, 15 pages, 2018. https://doi.org/10.1155/2018/9216193.
\bibitem{2018Qiu-Xiang} Li Qiu-xiang, Zhang Yu-hao, and Huang Yi-min, “The Complexity Analysis in Dual-Channel Supply Chain Based on Fairness Concern and Different Business Objectives,” \emph{Complexity}, vol. 2018, Article ID 4752765, 13 pages, 2018. https://doi.org/10.1155/2018/4752765.
\bibitem{2019Pathak} Pathak, U., Kant, R. , Shankar, R. OPSEARCH (2019). https://doi.org/10.1007/s12597-019-00421-z
\bibitem{2019Giri} B.C. Giri, S.K. Dey, Game theoretic analysis of a closed-loop supply chain with backup supplier under dual channel recycling, \emph{Computers \& Industrial Engineering}, Volume 129, 2019, Pages 179-191, ISSN 0360-8352, https://doi.org/10.1016/j.cie.2019.01.035. 
\bibitem{2019Hosseini-Motlagh} Seyyed-Mahdi Hosseini-Motlagh, Mina Nouri-Harzvili, Tsan-Ming Choi, Samira Ebrahimi, Reverse supply chain systems optimization with dual channel and demand disruptions: Sustainability, CSR investment and pricing coordination, \emph{Information Sciences}, Volume 503, 2019, Pages 606-634, ISSN 0020-0255, https://doi.org/10.1016/j.ins.2019.07.021.
\bibitem{2019Dai} Lufeng Dai, Xifu Wang, Xiaoguang Liu, and Lai Wei, “Pricing Strategies in Dual-Channel Supply Chain with a Fair Caring Retailer,” \emph{Complexity}, vol. 2019, Article ID 1484372, 23 pages, 2019. https://doi.org/10.1155/2019/1484372.
\bibitem{2019Qiu-Xiang1} Li, Q.; Chen, X.; Huang, Y. The Stability and Complexity Analysis of a Low-Carbon Supply Chain Considering Fairness Concern Behavior and Sales Service. \emph{Int. J. Environ. Res. Public Health} 2019, 16, 2711.
\bibitem{2019Qiu-Xiang2} Li, Q.; Shi, M.; Deng, Q.; Huang, Y.-M. The Complexity Entropy Analysis of a Supply Chain System Considering Recovery Rate and Channel Service. \emph{Entropy} 2019, 21, 659.

\end{thebibliography}
\end{document}